%\documentclass[12pt,a4paper]{report}
%\usepackage{setspace}

%\renewcommand{\baselinestretch}{1.5}
%------------------------------------------------------
\documentclass[a4paper,14pt]{article}

\usepackage[english]{babel}
\usepackage[utf8]{inputenc}
\usepackage{graphicx}
\usepackage{amsmath}
\usepackage{fancyhdr}
\usepackage{setspace}
\usepackage{listings}
\lstset{language=XML}
\renewcommand{\baselinestretch}{1.5}
\usepackage[
   top=3cm,
   right=2cm,
   left=2cm,
   bottom=3cm
]{geometry}

\usepackage{hyperref}

\pagestyle{fancy}
        \fancyhead[L]{\itshape{Kiruna Summer School 2014 Initiative}}
        %\fancyhead[R]{\itshape{[TEAM NAME]}}
\title{}


\begin{document}

\thispagestyle{empty}
\begin{center}


%\Large{\textbf{The CanSat project}}
\normalsize{\textbf{Kiruna Summer School (in winter) Initiative}}
\vspace{7cm}

%\Large{\bfseries{Low-cost ground station networks}\par}
%\Large{\bfseries{Seminar on formation flying}\par}

\vspace {1cm}
June, 2013 \\
\end{center}
%\vspace{7cm}
%\end{center}
%Author: Oliver Porges, Space Master round 7\\
%Supervisor: Prof.Dr. Marco Schmidt, Head of space activities

 
%\emph{Supervised by:}  Ing. Michal \textsc{Reinstein} Ph.D.\\

%------------------------------------------------------------------------
\newpage
\tableofcontents
\newpage
\vspace{7cm}
\section{Introduction}

There are numerous summer schools events through out the year. They are organized by educational and research institutions
seeking the following goals,
\begin{itemize}
\item Motivate perspective students
\item Networking between students and professionals in the field
\item Publicity of organizing institution
\end{itemize}

\section{Motivation}
The initiative comes from current and former Space Master students. We would like to show gratitude for the education and 
opportunities that were provided to us by LTU and Kiruna Campus. As there are 8 rounds of Space Master alumni
around the world we are forming a strong academic community, outreaching to all continents and across various
fields. Those who are pursuing careers in academics are working on state-of-the art research. Forming the feedback
of alumnis back to the program can open new opportunities for future graduates and rise the prestige of the program. 

Kiruna Summer School should be one of the events bringing former graduates, professionals and future students together, 
attracting international attention to Kiruna Campus. The LTU/IRF Campus facilities provide perfect settings for such event.

\section{Structure}
Duration of the school can range from 5 to 10 days, depending on the funding. Summer school should consist of the following modules,
\begin{itemize}
\item Lectures \\ Given by professinals in the filed, they should introduce 
the students to the topic and bring them up to speed with current state of the art.
\item Workshops \\ The aim is to provide practical experience on the topics presented in the lectures 
and give hands on experience with real world application
\item Poster competition \\ Poster competitions is usually based on the previous work of the interested student
to demonstrate his/her interest in the scope of the Sumemr School.
\item Workshop competition \\ Worksops ideally follow a certain goal which can be evaluated at the end. 
A team competition would be a motivation throughout the event. 
\end{itemize}

\section{Milestones}
\begin{enumerate}
\item Pinpoint possible dates of the event
\item List of interested organizers
\item Define a theme for the event
\item Approach relevant lecturers form the curriculum
\item Finalize funding - IEEE? LKAB? IRF? DLR? EU? 
\end{enumerate}


\section{Possible workshops}
Probably the best idea would be to use the Cansat class hardware from Professor Montenegro's group
at University of Wurzburg. We could focus the workshop curriculum around programming a micro-controller, 
signal processing, attitude control, image processing and communications. 


\end{document}























